\documentclass[prfluids]{revtex4-2}

\usepackage{amsmath}
\usepackage{mathrsfs}
\usepackage{graphicx}
\usepackage[colorlinks=true, allcolors=blue]{hyperref}
\usepackage{setspace}
\usepackage{xcolor}
\usepackage[T1]{fontenc}
\usepackage{tgheros}
\usepackage{helvet}
\usepackage{upgreek}
\usepackage[paper=A4, DIV=10]{typearea}

\renewcommand*\familydefault{\sfdefault}


\renewcommand*{\vec}[1]{\boldsymbol{#1}}
\newcommand{\paramF  }{\mathsf{\color{red}F}}
\newcommand{\paramOm}{        {\color{red}\omega}}

% \newcommand{\zeroplu}{\text{O}^+}
% \newcommand{\zeromin}{\text{O}^-}
% \newcommand{\oneplus}{\text{I}^+}
% \newcommand{\onemins}{\text{I}^-}
% \newcommand{\twoplus}{\text{II}^+}
% \newcommand{\twomins}{\text{II}^-}
% \newcommand{\treplus}{\text{III}^+}
% \newcommand{\tremins}{\text{III}^-}
% \newcommand{\Qplus}[1]{\text{Q}^{+}_{#1}}
% \newcommand{\Qmins}[1]{\text{Q}^{-}_{#1}}

\newcommand{\zeroplu}{\text{O}^+}
\newcommand{\zeromin}{\text{O}^-}
\newcommand{\oneplus}{\text{I}^+}
\newcommand{\onemins}{\text{I}^-}
\newcommand{\twoplus}{\text{II}^+}
\newcommand{\twomins}{\text{II}^-}
\newcommand{\treplus}{\text{III}^+}
\newcommand{\tremins}{\text{III}^-}
\newcommand{\Qplus}[1]{\text{Q}^{+}_{#1}}
\newcommand{\Qmins}[1]{\text{Q}^{-}_{#1}}


\begin{document}
\doublespacing

\section*{Linear stability analysis}

\subsubsection*{The approach of Kumar \& Tuckerman~\cite{Kumar_1994}}

We briefly describe an implementation of Kumar \& Tuckerman~\cite{Kumar_1994}.
We consider two layers of immiscible and incompressible fluids: fluid 1 (density
$\rho_1$, viscosity $\mu_1$) is defined between $z=z_0$ and $z=z_1$; fluid 2
(density $\rho_2$, viscosity $\mu_2$) is defined between $z=z_1$ and $z=z_2$.
The interface between fluids 1 and 2 is $\zeta$ (surface tension coefficient
$\gamma$). The equations of motion are given by
\begin{subequations}
  \begin{align}
    \label{eq1a}
    \rho_j \left[
      \partial_j + (\vec{U}_j\cdot\nabla) \right]\vec{U}_j &=
    -\nabla(P_j) + \mu_j\nabla^2\vec{U}_j
    - \rho_j G(t) \vec{e}_z
    \\
    \label{eq1b}
    \nabla\cdot\vec{U}_j &= 0
  \end{align}
\end{subequations}
where $\vec{U}=(u_j,v_j,w_j)$ and $G(t)=g (1 + \paramF\cos(\omega t))$.

We will consider a horizontally infinite plane, whose normal modes are
trigonometric functions, e.g. $\sin (\vec{k}\cdot\vec{x} )$. The horizontal wave
number $\vec{k}=k_x\vec{i} + k_y\vec{j}$, where $k^2 = k_x^2 + k_y^2$, can take
any real value. We can expand the fields in terms of horizontal normal modes of
the Laplacian since the form of the equations is such that each mode is
decoupled from the others. This is the approach followed by Benjamin \&
Ursell~\cite{Benjamin1954} for the ideal fluid case, and it remains valid for
the viscous fluid equations in the present case. We now simply replace 
\begin{subequations}
  \begin{align}
    \label{eq2a}
    w_j^*(\vec{x},z,t) &= \sin(\vec{k}\cdot\vec{x}) w_j (z,t) 
    \\
    \label{eq2b}
    \zeta^*(\vec{x},t) &= \sin(\vec{k}\cdot\vec{x}) \zeta (t) 
  \end{align}
\end{subequations}
and the differential operator $\nabla^2_H$ by the algebraic one $-k^2$

The complete linear stability problem reads
\begin{subequations}
  \begin{align}
    \label{eq3a}
    [\partial_t - \nu_1 (\partial_{zz} - k^2)](\partial_zz - k^2) w_1 &= 0, && \text{ for } -h_1 \leq z < 0
    \\
    \label{eq3b}
    [\partial_t - \nu_2 (\partial_{zz} - k^2)](\partial_zz - k^2) w_2 &= 0, && \text{ for } 0 \leq z < h_2
  \end{align}
  The boundary conditions at the two plates are given by
  \begin{align}
    \label{eq3c}
    w_1 &= 0, && \text{ for } z = -h_1
    \\
    \label{eq3d}
    w_2 &= 0, && \text{ for } z = h_2
    \\
    \label{eq3e}
    \partial_z w_1 &= 0, && \text{ for } z = -h_1
    \\
    \label{eq3f}
    \partial_z w_2 &= 0, && \text{ for } z = h_2
  \end{align}
  and the conditions at the interface are
  \begin{align}
    \label{eq3g}
    w_1 - w_2 &= 0,
    \\
    \label{eq3h}
    \partial w_1 - \partial w_2 &= 0,
    \\
    \label{eq3i}
    \nu_1 (\partial_{zz} + k^2) w_1 - \nu_2 (\partial_{zz} + k^2) w_2 &= 0,
    \\
    \label{eq3j}
    [\rho_1 \partial_t - \nu_1 (\partial_{zz} - k^2) + 2 \nu_1 k^2] \partial_z w_1
    -[\rho_2 \partial_t - \nu_2 (\partial_{zz} - k^2) + 2 \nu_2 k^2] \partial_z w_2
    &= - [(\rho_1-\rho_2)g(t) - \sigma k^2] k^2 \zeta,
  \end{align}
  The kinematic condition at the interface reads
  \begin{align}
    \label{eq3k}
    \partial \zeta - w  \vert_{z=0} = 0
  \end{align}
\end{subequations}
The above set of equations \eqref{eq3a}-\eqref{eq3k} constitute the full
hydrodynamic system, which we shall refer to as FHS.

In standard fashion, we search for solutions of Floquet form, i.e.,
$w_j(t)=w_j(t+nT)$ with integer $n$ and $T=2\pi/\omega$, 
\begin{align}
  \label{eq4}
  w_j (z,t) &= e^{(i\alpha+\lambda)t} \tilde{w}_j (z, t \mod T)
\end{align}
where $i\alpha+\lambda$ is the Floquet exponent and
$e^{(i\alpha + \lambda)T}$ is the Floquet multiplier. The function is
periodic in time with period $T$, and may therefore be expanded in the
Fourier series
\begin{align}
  \label{eq5}
  \tilde{w}_j (z, t \mod T) = \sum_{n=-\infty}^{\infty} w_{jn}(z) e^{in\omega t}
\end{align}
The Floquet multipliers are eigenvalues of a real mapping: this implies that
they are either real or complex-conjugate pairs. In addition, $\alpha$ is
defined only modulo $\omega$, since integer multiples of $\omega$ may be
absorbed into $\tilde{w}_j$. Hence, we restrict consideration to the range $0
\leq \alpha < \omega$. The two cases $\alpha = 0$ and $\alpha = \omega/2$ are
called harmonic and subharmonic, respectively, and correspond to positive or
negative real Floquet multipliers, whereas $0 < \alpha < \omega/2$ corresponds
to a complex Floquet multiplier.

The relationship between Fourier modes with positive and negative $n$ depends on
the value of $\alpha$. In the harmonic and subharmonic cases, $\tilde{w}_j$ must
obey reality conditions $w_{j,-n} = w_{j,n}^*$ (harmonic) or $w_{j,-n} =
w_{j,n-1}^*$ (harmonic) (subharmonic), so that the series may be rewritten in
terms only of non-negative Fourier indices. Only the harmonic and subharmonic
cases are relevant to this linear stability analysis : complex Floquet
multipliers are always of magnitude less than or equal to one, and hence do not
correspond to growing solutions. 

The interface position is expanded in the same way:
\begin{subequations}
  \begin{align}
    \label{eq6a}
    \zeta(t) &= e^{(i\alpha+\lambda)t} \tilde{\zeta} (t \mod 2\pi/\omega)
    \\
    \label{eq6b}
    \tilde{\zeta} (t \mod 2\pi/\omega) &= \sum_{n=-\infty}^{\infty} \zeta_{n}(z) e^{in\omega t}
  \end{align}
\end{subequations}
with the same reality conditions as for $w_j$.

Equations \eqref{eq3g} and \eqref{eq3k} imply
\begin{align}
  \label{eq7}
  w_{1n} (z=0) = w_{2n} (z=0) = [i(\alpha + n\omega) + \lambda] \zeta_n
\end{align}
Also, \eqref{eq3a} and \eqref{eq3b} imply
\begin{subequations}
  \begin{align}
    \label{eq8a}
    [i(\alpha + n\omega) + \lambda - \nu_1 (\partial_{zz} - k^2)](\partial_zz - k^2) w_{1n} &= 0, && \text{ for } -h_1 \leq z < 0
    \\
    \label{eq8b}
    [i(\alpha + n\omega) + \lambda - \nu_2 (\partial_{zz} - k^2)](\partial_zz - k^2) w_{2n} &= 0, && \text{ for } 0 \leq z < h_2
  \end{align}
\end{subequations}
with solutions
\begin{align}
  \label{eq9}
  w_{jn}(z) &=
  \left\lbrace\begin{matrix}
    a_{jn}e^{kz} &+& b_{jn}e^{-kz}
    &&&&& \text{ for } &\nu_j = 0 
    \\
    a_{jn}e^{kz}    &+& b_{jn}e^{-kz}&+&
    c_{jn}e^{q_{jn}z} &+& d_{jn}e^{-q_{jn}z}
    & \text{ for } &\lambda + i (\alpha + n) \neq 0 
    \\
    a_{jn}e^{kz}  &+& b_{jn}e^{-kz}&+&
    c_{jn}ze^{kz} &+& d_{jn}ze^{-kz}
    & \text{ for } & \lambda + i (\alpha + n) = 0 \\
  \end{matrix}\right.
\end{align}
where 
\begin{align}
  \label{eq10}
  q^2_{jn} = k^2 + \frac{i(\alpha + n\omega) + \lambda}{\nu_j}
\end{align}
with the convention that $q^2_{jn}$ is the root with positive real part.

Introducing these expansions into the full hydrodynamic system, we need to solve
a linear equation system for each Fourier mode to find the coefficients $a_{jn}$
to $d_{jn}$. 
Having solved the coefficients, we are left with the pressure jump
condition to couple the different modes
\begin{align}
  \label{eq11}
  \Delta [\rho\lbrace \lambda + i (\alpha + n\omega)\rbrace + 3 \mu k^2] \partial_z w_n -
  \Delta\mu\partial_{zzz} w_n +
  [\Delta\rho g - \gamma k^2] k^2 \zeta_n
  &= \frac{\Delta\rho k^2}{2} a (\zeta_{n+1}+\zeta_{n-1})
\end{align}
where $\partial_z w$ reads
\begin{align}
  \label{eq12}
  \partial_zw_{jn} &=
  \left\lbrace\begin{array}{rrrrrrrrrr}
    a_{jn}ke^{kz} &-& b_{jn}ke^{-kz}
    &&&&& \text{ for } &\nu_j = 0 \\
    a_{jn}ke^{kz}    &-& b_{jn}ke^{-kz}&+&
    c_{jn}q_{jn} e^{q_{jn}z} &-& d_{jn}q_{jn} e^{-q_{jn}z}
    & \text{ for } &\lambda + i (\alpha + n) \neq 0 \\
    a_{jn}ke^{kz}  &-& b_{jn}ke^{-kz}&+&
    c_{jn}(1 + kz)e^{kz} &+& d_{jn}(1 - kz)e^{-kz}
    & \text{ for } & \lambda + i (\alpha + n) = 0 \\
  \end{array}\right.
\end{align}
and $\partial_{zzz} w$ reads
\begin{align}
  \label{eq13}
  \partial_{zzz}w_{jn} &=
  \left\lbrace\begin{array}{rrrrrrrrrr}
    a_{jn}k^3e^{kz}    &-& b_{jn}k^3e^{-kz}&+&
    c_{jn}q_{jn}^3e^{q_{jn}z} &-& d_{jn}q_{jn}^3e^{-q_{jn}z}
    & \text{ for } &\lambda + i (\alpha + n) \neq 0 \\
    a_{jn}k^3e^{kz}  &-& b_{jn}k^3e^{-kz}&+&
    c_{jn}(3k^2 + k^3z)e^{kz} &+& d_{jn}(3k^2 - k^3z)e^{-kz}
    & \text{ for } & \lambda + i (\alpha + n) = 0 \\
  \end{array}\right.
\end{align}
such that equation \eqref{eq11} can be written as
\begin{eqnarray}
  A_n \zeta_n &=& a (\zeta_{n+1}+\zeta_{n-1}).
\end{eqnarray}

Finally, taking into account the reality conditions, $\zeta_{-1}=\zeta_{1}^*$
(harmonic) and $\zeta_{-1}=\zeta_{0}^*$ (sub-harmonic), and truncating the
Fourier series to $0 \leq n \leq N$, we obtain a (2N+1)x(2N+1) eigenvalue
problem $\textbf{A}\boldsymbol{\zeta}=a\mathbf{B}\boldsymbol{\zeta}$,
\begin{eqnarray}
  \underbrace{\left[\begin{matrix}
    \begin{matrix}
    \Re(A_0) & -\Im(A_0) \\
    \Im(A_0) &  \Re(A_0)
    \end{matrix}
    \\&
    \begin{matrix}
      \Re(A_1) & -\Im(A_1) \\
      \Im(A_1) &  \Re(A_1)
    \end{matrix}
    \\&&
    \ddots
    \\&&&
    \begin{matrix}
      \Re(A_N) & -\Im(A_N) \\
      \Im(A_N) &  \Re(A_N)
    \end{matrix}
  \end{matrix}\right]}_{\textbf{A}}
  \underbrace{
  \left[\begin{matrix}
    \Re(\zeta_0) \\ \Im(\zeta_0) \\
    \Re(\zeta_1) \\ \Im(\zeta_1) \\ \vdots \\
    \Re(\zeta_N) \\ \Im(\zeta_N)
  \end{matrix}\right]
  }_{\boldsymbol{\zeta}}
   = a\textbf{B}
   \underbrace{
  \left[\begin{matrix}
    \Re(\zeta_0) \\ \Im(\zeta_0) \\
    \Re(\zeta_1) \\ \Im(\zeta_1) \\ \vdots \\
    \Re(\zeta_N) \\ \Im(\zeta_N)
  \end{matrix}\right]
  }_{\boldsymbol{\zeta}}
\end{eqnarray}
where the structure of $\textbf{B}$ depends on $\alpha$
\begin{eqnarray}
\textbf{B} =
\left[\begin{matrix}
  0 &0 &2 &0 &0 & 0 &\cdots \\
  0 &0 &0 &0 &0 & 0 &\cdots \\
  1 &0 &0 &0 &1 & 0 &\cdots \\
  0 &1 &0 &0 &0 & 1 &\cdots \\
  0 &0 &1 &0 &0 & 0 &\cdots \\
  \vdots & \vdots & \vdots & \vdots & \vdots &\vdots & \ddots
\end{matrix}\right]
\text{ (harmonic) }
\quad\quad
\textbf{B} =
\left[\begin{matrix}
  1 &0 &1 &0 &0 & 0 &\cdots \\
  0 &-1&0 &1 &0 & 0 &\cdots \\
  1 &0 &0 &0 &1 & 0 &\cdots \\
  0 &1 &0 &0 &0 & 1 &\cdots \\
  0 &0 &1 &0 &0 & 0 &\cdots \\
  \vdots & \vdots & \vdots & \vdots & \vdots &\vdots & \ddots
\end{matrix}\right]
\text{ (sub-harmonic) }.
\end{eqnarray}


\newpage
\clearpage
\KOMAoptions{paper={landscape}}
\recalctypearea

\paragraph{Viscous fluids for $\sigma + i(\alpha + \omega n)\neq0$ (finite depth)}

\begin{footnotesize}\begin{eqnarray*}
  \left[\begin{array}{r r r r r r r r }
    \oneplus & \onemins  & \Qplus{11} & \Qmins{11}
    \\
    \oneplus & \onemins  & \Qplus{11} & \Qmins{11} & -\oneplus & -\onemins  &
    -\Qplus{21} & -\Qmins{21}
    \\
    k\oneplus & -k\onemins  & q_1\Qplus{11} & -q_1\Qmins{11} &
    -k\oneplus & k\onemins  & -q_2\Qplus{21} & q_2\Qmins{21}
    \\
    2k^2\mu_1\oneplus & 2k^2\mu_1\onemins  & (k^2+q_1^2)\mu_1\Qplus{11} &
    (k^2+q_1^2)\mu_1\Qmins{11} &
    -2k^2\mu_2\oneplus & -2k^2\mu_2\onemins  & -(k^2+q_2^2)\mu_2\Qplus{21} &
    -(k^2+q_2^2)\mu_2\Qmins{21}
    \\
    \zeroplu & \zeromin  & \Qplus{10} & \Qmins{10}
    \\
    k\zeroplu & -k\zeromin  & q_1\Qplus{10} & -q_1\Qmins{10}
    \\
    &&&& \twoplus & \twomins  & \Qplus{22} & \Qmins{22}
    \\
    &&&& k\twoplus & -k\twomins  & q_2\Qplus{22} & -q_2\Qmins{22}
  \end{array}\right]
  \left[\begin{matrix}
    a_1 \\ b_1 \\ c_1 \\ d_1 \\
    a_2 \\ b_2 \\ c_2 \\ d_2
  \end{matrix}\right]
  =
  \left[\begin{matrix}
    \sigma + i(\alpha + \omega n) \\ 0 \\ 0 \\ 0
    \\ 0 \\ 0 \\ 0 \\ 0
  \end{matrix}\right]
  \quad\text{ with }\quad \text{Q}^\pm_{ij} \equiv \exp{(\pm q_{in} z_j)}
\end{eqnarray*}\end{footnotesize}

\paragraph{Viscous fluids for $\sigma + i(\alpha + \omega n)=0$ (finite depth)}

\begin{footnotesize}\begin{eqnarray*}
  \left[\begin{array}{r r r r r r r r }
         \oneplus &   \onemins &      z_1\oneplus &      z_1\onemins \\
         \oneplus &   \onemins &      z_0\oneplus &      z_0\onemins &
         -\oneplus &  -\onemins &     -z_0\oneplus &     -z_0\onemins \\
         k\oneplus & -k\onemins & (1+kz_0)\oneplus & (1-kz_0)\onemins &
        -k\oneplus &  k\onemins &-(1+kz_0)\oneplus &-(1-kz_0)\onemins \\
      2k\mu_1        \oneplus & 2k^2\mu_1         \onemins &
     (2k^2z_0 + 2k)\mu_1 \oneplus & (2k^2z_0 - 2k) \mu_1\onemins &
     -2k^2\mu_2         \oneplus & -2k^2\mu_2         \onemins &
     -(2k^2z_0 + 2k)\mu_2 \oneplus & -(2k^2z_0 - 2k) \mu_2\onemins \\
         \zeroplu &   \zeromin &      z_0\zeroplu &      z_0\zeromin \\
        k\zeroplu & -k\zeromin & (1+kz_0)\zeroplu & (1-kz_0)\zeromin \\
    &&&& \twoplus &   \twomins &      z_2\twoplus &      z_2\twomins \\
    &&&&k\twoplus & -k\twomins & (1+kz_2)\twoplus & (1-kz_2)\twomins 
  \end{array}\right]
  \left[\begin{matrix}
    a_1 \\ b_1 \\ c_1 \\ d_1 \\
    a_2 \\ b_2 \\ c_2 \\ d_2
  \end{matrix}\right]
  =
  \left[\begin{matrix}
    0 \\ 0 \\ 0 \\ 0
    \\ 0 \\ 0 \\ 0 \\ 0
  \end{matrix}\right]
\end{eqnarray*}\end{footnotesize}


\paragraph{Viscous fluids for $\sigma + i(\alpha + \omega n)\neq0$ (deep layers)}

\begin{footnotesize}\begin{eqnarray*}
  \left[\begin{array}{r r r r r r r r }
    \oneplus & \Qplus{11}
    \\
    \oneplus & \Qplus{11} & -\onemins & -\Qmins{21}
    \\
    k\oneplus & q_1\Qplus{11} & k\onemins &q_2\Qmins{21}
    \\
    2k^2\mu_1\oneplus & (k^2+q_1^2)\mu_1\Qplus{11} &
    -2k^2\mu_2\onemins  & -(k^2+q_2^2)\mu_2\Qmins{21}
  \end{array}\right]
  \left[\begin{matrix}
    a_1 \\ c_1 \\ b_2 \\ d_2
  \end{matrix}\right]
  =
  \left[\begin{matrix}
    \sigma + i(\alpha + \omega n) \\ 0 \\ 0 \\ 0
  \end{matrix}\right]
  \quad\text{ with }\quad \text{Q}^\pm_{ij} \equiv \exp{(\pm q_{in} z_j)}
\end{eqnarray*}\end{footnotesize}

\paragraph{Viscous fluids for $\sigma + i(\alpha + \omega n)=0$ (deep layers)}

\begin{footnotesize}\begin{eqnarray*}
  \left[\begin{array}{r r r r r r r r }
    \oneplus &   z_1\oneplus \\
    \oneplus &   z_0\oneplus & -\onemins & -z_0\onemins \\
    k\oneplus & (1+kz_0)\oneplus & k\onemins & -(1-kz_0)\onemins \\
     2k^2\mu_1 \oneplus & (2k^2z_0 + 2k)\mu_1 \oneplus & 
    -2k^2\mu_2 \onemins & -(2k^2z_0 - 2k) \mu_2\onemins \\
  \end{array}\right]
  \left[\begin{matrix}
    a_1 \\ c_1 \\ b_2 \\ d_2
  \end{matrix}\right]
  =
  \left[\begin{matrix}
    0 \\ 0 \\ 0 \\ 0
  \end{matrix}\right]
\end{eqnarray*}\end{footnotesize}


\paragraph{Ideal fluids}
\begin{footnotesize}\begin{eqnarray}
  \left[\begin{array}{r r r r}
    \oneplus & \onemins  &                        \\    
    \zeroplu & \zeromin  &                        \\    
             &           &  \twoplus &  \twomins  \\
    \oneplus & \onemins  & -\oneplus & -\onemins
  \end{array}\right]
  \left[\begin{matrix}
    a_1 \\ b_1 \\ a_2 \\ b_2
  \end{matrix}\right]
  =
  \left[\begin{matrix}
    \sigma + i(\alpha + \omega n) \\ 0 \\ 0 \\ 0
  \end{matrix}\right]
  \quad\text{ with }\quad
  \text{O}^\pm \equiv \exp^{(\pm kz_0)}, \quad
  \text{I}^\pm \equiv \exp^{(\pm kz_1)}, \quad
  \text{II}^\pm \equiv \exp^{(\pm kz_2)}.
\end{eqnarray}\end{footnotesize}


\bibliographystyle{plain}
\bibliography{../../biblio/faramix}

\end{document}